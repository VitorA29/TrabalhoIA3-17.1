\textit{O que é pensado pelas pessoas} sempre foi uma informação importante para seres humanos para o processo de tomada de decisão. Com o advento da \emph{World Wide Web}, cresceu o acesso à quantidade de opiniões e experiências sobre determinados assuntos que são de pessoas que não conhecemos e nem são profissionais especialistas no assunto. Dessa forma, é possível obter informações de pessoas com os mais variados sentimentos acerca de algum assunto.

Nesse espectro, surge a área de análise de sentimentos (ou mineiração de opiniões), que é responsável por fazer o processamento de linguagem natural, usando táticas de análise textual e linguística computacional a fim de identificar, extrair e estudar opiniões, estados afetivos e informação subjetiva. Dessa forma, é possível extrair opiniões de consumidores acerca de um determinado produto, por exemplo. Tal mineiração é extremamente útil, pois como é visto em  \cite{pang2008opinion}, influencia bastante em tópicos como a aquisição de serviços: a cada 2000 americanos, dentre os leitores de resenhas on-line de restaurantes, hotéis e outros serviços, como viagens, escolas, médicos e cursos, de 73\% à 87\% dos entrevistados disseram que tais resenhas tiveram uma influência significante na aquisção desses serviços \cite{zhu2010impact}.

Tal abordagem também é útil para outras finalidades: além da compra de serviços e produtos, as revisões de outros usuários online também são úteis na busca de opiniões políticas (tanto acerca de empresas e organizações quanto acerca de políticos): muitas pessoas buscam atualmente informações de outras acerca de políticos, por exemplo, para confirmar se a opinião dele é condizente com a sua, ou até mesmo buscam na internet opiniões que divergem das suas a fim de enriquecer o debate \cite{gil2009weblogs}.

Com o advento de plataformas na \emph{web}, tais como blogs, fóruns de discussão, redes \emph{peer-to-peer} e outros tipos de \emph{social media}, tais como o \emph{Facebook} e o \emph{Twitter}. consumidores têm uma quantidade de informação e uma facilidade de expor sua opinião sem precedentes, sejam elas negativas ou positivas. sobre qualquer produto ou serviço. Nesse âmbito, grandes companhias (bancos, restaurantes, agências de viagem, redes de \emph{fast-food} e muitas outras companhinhas dos mais diversos ramos) buscam ler desse "apelo" informações relevantes para satisfazer as opiniões dos potenciais clientes; em outras palavras, essas opiniões podem exercer uma influência enorme na formação de opiniões de outros usuários, formando a "lealdade" à marca, o público consumidor, podendo alavancar ou condenar um determinado produto ou até mesmo a imagem de uma empresa \cite{hoffman2008online}.
