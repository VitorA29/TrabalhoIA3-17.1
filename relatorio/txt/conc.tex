Este trabalho apresentou uma comparação de técnicas de aprendizado de máquina no contexto de análise de sentimentos através de \textit{tweets}. Dentre as técnicas utilizadas destacam-se o desempenho dos classificadores \textit{Decision Tree} e \textit{Random Forest}. Neles foram observados uma acurácia maior quando comparados aos classificares \textit{SVM} e \textit{Naive Bayes}.

Por sua vez, o \textit{Grid Search}, ao realizar a otimização de hiperparametros no contexto do classificador \textit{Naive Bayes} para o caso de execução quaternária, apresentou uma melhora de 18\% da acurácia.

Por fim, este trabalho possibilitou a utilização da ferramenta para aprendizado de máquina \textit{Scikit-learn} e conhecimento de novas técnicas e aplicação prática delas. A maior dificuldade enfrentada foi o ajuste dos \textit{inputs} dos classificadores e o entendimento e utilização das técnicas de \textit{Grid Search}, \textit{PCA} e \textit{RFA}.